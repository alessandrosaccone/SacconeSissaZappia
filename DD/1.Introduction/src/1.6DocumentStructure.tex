\subsection{Document Structure}

The following of the Design Document for the \app project is divided into four major sections, which aim at describing the overall design of the system in order to support the development and implementation of the final product.

\textbf{Section 2, Architectural Design}, is the most design-relevant one. Its main objective is to provide the reader with a series of views or structure of the software architecture that has been selected for \app. Indeed, section 2 is further divided into:
\begin{itemize}
	\item Overview: provides a high-level description of the most important components of the system and their interactions (in an informal notation).
	\item Component view: this section provides the first structure of the software architecture, which is the Component \& Connector structure, useful for illustrating the relevant components of the system from a dynamic perspective and the way they collaborate to achieve the final goals. In this section, UML components diagrams are largely employed to convey these ideas and notions.
	\item Deployment view: this section is about another structure of the software architecture, related to the deployment of \app. The most important objective of this chapter is to show the mapping between the software components of \app and the hardware devices that will physically execute the app. UML deployment diagrams are a great tool to unpack this topic.
	\item Runtime view: this section employs sequence diagrams in order to explain the flow of events and interactions within the system's components, in a consistent way with the previous chapters.
	\item Component interfaces: in here it is possible to provide a complete specification of the most important methods and functions that each interface exposed by the various components of the system must show. 
	\item Selected architectural styles and patterns: a revision of the main architectural styles and patterns with a more in detail explanation of the reasons why they have been chosen for this project.
	
\end{itemize}

\textbf{Section 3, User Interface Design}, is about user interfaces (UI). 
More specifically, this section is concerned with giving some guidelines to UI designers on how of the final application should look like (color themes, placement of most relevant UI items...), as well as on the logical role that these interfaces have in the development (what functionalities they provide to the user).

Moving on, \textbf{section 4 (Requirement Traceability)} is dedicated to a matrix that shows how the requirements that have been drawn and derived for \app map onto the components that have been highlighted in the previous sections of the document.

Finally, \textbf{section 5 (Implementation, Integration and Test Plan)} is concerned with illustrating the implementation strategy adopted (order of implementation of components), the integration strategy (how to integrate new sub-components into the application under-development), and the test strategy for the integration of different components in the system.