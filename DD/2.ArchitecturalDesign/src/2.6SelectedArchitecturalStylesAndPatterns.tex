\subsection{Selected Architectural Styles and Patterns}

The choice of this architectural style is due to many factors, including:

\begin{itemize}
	\item \textit{Scalability}: Microservices can be scaled independently, allowing for scaling out sub-services without scaling out the entire system. This leads to the versatility of the application.
	
	\item \textit{Fault Tolerance}: Unlike the monolithic approach, which has many inter-dependencies creating a single point of failure, in this approach, the application can remain mostly unaffected by the failure of a single module.
	
	\item \textit{Deployment and Productivity}: Microservices enable continuous integration and delivery, making it easy to test new ideas, suiting Agile and DevOps working methodologies. Furthermore, it makes it easier to split the work between team members: each team member is responsible for a particular service, resulting in a smart, productive, cross-functional team where the speed of development is largely improved.
	
	\item \textit{Continuous Delivery}: Microservices enable continuous delivery, meaning your software can be modified and delivered to your client base frequently and easily due to its automated nature.
	
	\item \textit{Maintainability}: Benefits of Microservices Architecture include less energy spent on understanding separate pieces of software or worrying about how a bug fix will affect other parts of the product.
\end{itemize}