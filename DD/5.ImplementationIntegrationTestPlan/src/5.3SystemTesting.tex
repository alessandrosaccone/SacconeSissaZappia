
\subsection{System Testing}

System testing is a critical phase where the complete integrated system is put to the test. The testing environment should closely mirror the actual production setup to ensure a complete evaluation of the software's final version. To increase the chances of a successful initial software release, we perform both functional testing and non-functional one (performance, load and stress testing). 

Functional testing checks if the software meets the requirements outlined in the CKB's Requirements Analysis and Specification Document (RASD), as described, for example, in the use cases: after implementing the software, it's crucial to confirm that these requirements have been met.

Performance testing includes non-functional tests like assessing response time, throughput, and identifying any architectural issues; the goal is to determine if adjustments are needed in the architecture of the Document Design (DD) to meet the software's requirements. 

Load testing, a subset of performance testing, involves gradually increasing the workload or sustaining loading to ensure that the system can handle the expected number of users. 

Stress testing focuses on the system capacity of recovery after a failure, emphasizing availability over complete reliability, as requested for the software.

For test case generation in the CKB system, we use both manual and automatic methods. Manual testing helps identify specific situations, while automatic testing ensures a comprehensive assessment of various scenarios. To cover the entire system, automated testing employs either concolic execution, and fuzz testing, and search-based strategy.

It's essential to note that simpler tests, like those outlined as asserts in the Alloy model of the CKB system in the RASD, serve as a starting point for system testing.\

