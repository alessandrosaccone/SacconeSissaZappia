\section{Requirements}
	
	\subsection{Vocabulary}
	\begin{itemize}
		\item Students and Educators are the two possible Users
		\item The system to be developed can also be called application, software, platform
		\item Code kata battles are abbreviated as battles
		\item The consolidation stage is the optional phase of a battle lifecycle that comes after the submission deadline and it consists in a period of time in which the educator that created the battle is able to scrutinize the code solutions provided by the students and assign personal scores to those solutions
	\end{itemize}
	
	
	Let's start with the division of phenomena happening in the system and inside the environment surrounding the system. These lists of elements have to be interpreted simply as events occurring in the whole domain that is being analyzed in the document, so they have been stripped of any constraint (that will be better specified in the requirements section).
	
	
	
	\subsection{World phenomena}
	
	\begin{itemize}
		\item \textbf{Educator} wants to support students in bettering their software development skills
		\item \textbf{Educator} comes up with new code kata battles ideas
		\item \textbf{Educator} writes the textual description of a new battle s/he wants to create
		\item \textbf{Educator} designs the test cases and the build automation scripts of a new battle s/he wants to create
		\item \textbf{Student} wants to improve their software development skills
		\item \textbf{Student} forks the main GitHub repository that hosts a battle to upload his/her code solutions
		\item \textbf{Student} writes an automated workflow through GitHub Actions to setup the interaction between GitHub and the system
		\item \textbf{Student} writes the code solutions for a battles
		\item \textbf{Student} pushes on the forked branch of the GitHub repository hosting a battle the  code solution of his/her team, generating a new commit


	\end{itemize}
	
	\subsection{Shared phenomena}
	
	\subsubsection{World controlled}
	
	\begin{itemize}
		\item \textbf{User} signs up on the system for the first time and creates a new account
		\item \textbf{User} inserts credentials and logs in the system
		\item \textbf{Educator} creates new tournament(s) with a registration deadline
		\item \textbf{Educator} defines the badges (rewards) for the tournament s/he wants to create
		\item \textbf{Educator} grants permissions to create battles to other educators
		\item \textbf{Educator} creates battle(s) in a tournament
		\item \textbf{Educator} uploads on the system the textual description of the battle s/he wants to create
		\item \textbf{Educator} uploads on the system the test cases and build automation scripts to be employed in order to verify the students’ solutions for the battle s/he wants to create
		\item \textbf{Educator} inputs in the system the minimum and maximum number of students per group allowed for the battle s/he wants to create
		\item \textbf{Educator} inputs in the system the registration deadline of the battle s/he wants to create
		\item \textbf{Educator} inputs in the system the submission deadline for the students’ solutions of the battle s/he wants to create
		\item \textbf{Educator} decides whether to require a consolidation stage after the submission deadline of the battle s/he wants to create
		\item \textbf{Educator} selects which aspects (reliability, maintainability, security...) should be taken into account by the system when computing the score of the students' code solutions for the battle s/he wants to create
		\item \textbf{Educator} revises the code solutions of the teams participating in a battle and assigns a personal score to each one of them
		\item \textbf{Educator} closes a tournament
		\item \textbf{Student} opens the description page of a tournament
		\item \textbf{Student} subscribes to a tournament
		\item \textbf{Student} opens the description page of a battle
		\item \textbf{Student} joins a battle of a tournament either forming a group with other students or on his/her own
		\item \textbf{User} opens his/her personal profile on the platform
		\item \textbf{User} opens the personal profile of another user on the platform
		\item \textbf{User} opens the partial or final ranking of students in a tournament
		\item \textbf{User} opens the partial or final ranking of teams in a battle
		\item \textbf{GitHub} notifies the system every time a new commit is pushed on a repository that stores the code solutions for a battle
		
	\end{itemize}
	
	\subsubsection{Machine controlled}
	
	\begin{itemize}
		\item \textbf{The system} shows to a \textbf{user} the “sign up” interface
		\item \textbf{The system} shows to a \textbf{user} the "log in" interface
		\item \textbf{The system} shows to a \textbf{user} the initial page (home page) to a user
		\item \textbf{The system} shows to an \textbf{educator} the interface for creating a new tournament
		\item \textbf{The system} shows to an \textbf{educator} the interface for creating a new battle in an ongoing tournament
		\item \textbf{The system} shows to an \textbf{educator} the interface for assigning personal scores during the consolidation stage of a battle, associated with a timer for the available time to complete the operation
		\item \textbf{The system} shows to a \textbf{user} the list of tournaments available on the platform
		\item \textbf{The system} shows to a \textbf{user} the description page of a tournament
		\item \textbf{The system} shows to a \textbf{student} the interface for subscribing to a tournament
		\item \textbf{The system} shows to a \textbf{user} the list of battles within a tournament
		\item \textbf{The system} shows to a \textbf{user} the description page of a battle
		\item \textbf{The system} shows to a \textbf{student} the interface for joining a battle
		\item \textbf{The system} shows to a \textbf{user} his personal profile
		\item \textbf{The system} shows to a \textbf{user} the personal profile of another user on the platform
		\item \textbf{The system} shows to a \textbf{user} the partial and final ranking of students in a tournament
		\item \textbf{The system} shows to a \textbf{user} the partial and final ranking of teams in a battle
		\item \textbf{The system} notifies all \textbf{students} subscribed to the platform when a new tournament is created
		\item \textbf{The system} notifies all \textbf{students} subscribed to a tournament when a new battle is created
		\item \textbf{The system} notifies all \textbf{students} that participated in a battle when the final ranking of the battle is available
		\item \textbf{The system} notifies all \textbf{students} subscribed to a tournament when the tournament is closed
		\item \textbf{The system} creates a new \textbf{GitHub} repository to store the students’ code solutions for a battle
		\item \textbf{The system} sends the link of the GitHub repository (used to store the students' code solutions of a battle) to all the \textbf{students} subscribed to the battle
		\item \textbf{The system} shows and keeps updated the scores assigned to the students' code solutions for a battle
		\item \textbf{The system} assigns badges (rewards) to \textbf{students} when a tournament ends
	\end{itemize}
	
	\subsection{Machine phenomena}
	
	\begin{itemize}
		\item \textbf{The system} pulls new code solutions from GitHub
		\item \textbf{The system} runs build automation scripts and test cases on new code solutions
		\item \textbf{The system} calculates the amount of time passed between the start of a battle and the moment in which a new submission has been handed in by a student (through GitHub)
		\item The system runs static analysis on the new code solutions exploiting external static analysis tools
		\item \textbf{The system} calculates the score to be assigned to a new code solution
		\item \textbf{The system} computes the partial or final ranking of students for a tournament
		\item \textbf{The system} computes the partial or final ranking of teams for a battle
		\item \textbf{the system} computes the set of students that are eligible for receiving a badge (reward) at the end of a tournament
	\end{itemize}
	
	\subsection{Specifications on world and machine phenomena}
	\begin{itemize}
		\item From the description of the RASD document, it seems that:
		\begin{itemize}
			\item The educators have to first write down the description of the battle and the test cases and automation scripts and then upload everything on the system. Since the system is not a text editor, it is assumed these “events” are world only because they cannot be observed by the system, which only sees the upload phase, as stated later in the shared phenomena.
			\item The students have to first write the code and then upload the solution on the platform in order to test it. Again it is assumed that the application doesn’t work as a text editor but it only receives the source code files and runs the tests on them. Therefore the action of typing the code is a world-only phenomena, while the system sees the upload phase.
		\end{itemize} 
		\item In order to model GitHub interactions, it is just necessary to be consistent in the document. So let’s take as a decision that GitHub is an external actor of the world that takes part in the system. Therefore, the interaction GitHub-machine is part of the shared world:
		\begin{itemize}
			\item The fork of the main repository on GitHub is world-only because it is not observed by the machine. It will become a world assumption later on
			\item The action of writing the GitHub Action workflow is world only because the system doesn’t see the workflow
			\item GitHub is an actor of the world so in the world controlled phenomena also the notification to the system is inserted.
			\item The action of pulling the code from github and running the tests is internal to the machine
		\end{itemize}
		\item Further assumptions:
		\begin{itemize}
			\item It is considered as an assumption the fact that only the educator who first created the tournament has the right permissions to close the tournament
		\end{itemize}
		
	\end{itemize}
	
	\subsection{Goals and requirements per goal}
	The goals take the point of view of a single user of the application and the requirements listed are drawn from the shared phenomena and represents the ways the machine reaches the goal
	\begin{itemize}
		\item G1) \textbf{The system} allows \textbf{students} to practice and improve their coding skills with code kata battles
		\begin{itemize}
			\item R1.1) \textbf{The system} allows \textbf{students} that don't have an account to sign up for the first time on the platform
			\item R1.2) \textbf{The system} allows \textbf{students} that already have an account to log in the platform and see the home (initial) page
			\item R1.3) \textbf{The system} notifies to all \textbf{students} that have an account on the platform every time a new tournament is created
			\item R1.4) \textbf{The system} shows to a \textbf{student} the list of tournaments available on the platform, both the ones the student is already subscribed to and the ones s/he can subscribe to
			\item R1.5) \textbf{The system} allows \textbf{students} to join the tournament(s) they’re interested in before the registration deadline of the tournament expires
			\item R1.6) \textbf{The system} notifies all \textbf{students} subscribed to a tournament every time a new battle within that tournament is published
			\item R1.7) \textbf{The system} allows \textbf{students} subscribed to a tournament to join battle(s) within that tournament before the registration deadline of the battle expires
			\item R1.8) \textbf{The system} allows \textbf{students} to join a battle either on their own or creating groups with other students
			\item R1.9) \textbf{The system} creates a GitHub repository dedicated to a battle when the registration deadline for that battle expires, in order to store the students’ code solutions
			\item R1.10) \textbf{The system} notifies all the \textbf{students} subscribed to a battle when the GitHub repository for that battle has been created
			\item R1.11) \textbf{The system} listens for notifications from GitHub in order to understand when a new commit is performed by a student in any GitHub repository dedicated to an ongoing battle
			\item R1.12) \textbf{The system} pulls from the GitHub repositories related to ongoing battles the students' new code solutions, every time it receives a notification from GitHub of a new commit in those repositories
			\item R1.13) \textbf{The system} recalculates and then publishes the new score of a team's solution every time it receives a notification from GitHub of a new commit performed by a member of the team
			\item R1.14) \textbf{The system} doesn't take into account any new code solution for a battle after the submission deadline of that battle expires
			\item R1.15) \textbf{The system} calculates and publishes the final ranking of teams for a battle after the submission deadline of the battle expires
			\item R1.16) \textbf{The system} updates the ranking of students in a tournament every time a battle within that tournament ends
			\item R1.17) \textbf{The system} notifies all \textbf{students} subscribed to a tournament when the tournament is closed by the educator who created it
			\item R1.18) \textbf{The system} shows the final rank of students in a tournament when the tournament is closed by the educator that created it
		\end{itemize}
	\end{itemize}
	

	
	
	
	\begin{itemize}
		\item G2) \textbf{The system} provides to \textbf{educators} a platform to publish code kata battles and easily manage tournaments between students
		\begin{itemize}
			\item R2.1) \textbf{The system} allows \textbf{educators} that don’t have an account to sign up for the first time on the platform
			\item R2.2) \textbf{The system} allows \textbf{educators} that already have an account to log in the platform and see the home (initial) page
			\item R2.3) \textbf{The system} allows \textbf{educators} to create new tournaments
			\item R2.4) \textbf{The system} allows \textbf{educators} to set a registration deadline for the tournament(s) they want to create
			\item R2.5) \textbf{The system} allows \textbf{educators} to define customary badges (rewards) for the students participating in the tournament(s) at tournament creation time
			\item R2.6) \textbf{The system} allows the \textbf{educator} that created a new tournament to grant the permission of publishing battles within his/her tournament to other educators on the platform
			\item R2.7) \textbf{The system} allows \textbf{educators} to create new battle(s) in the tournaments they have the permissions to do so
			\item R2.8) \textbf{The system} allows \textbf{educators} to upload on the platform the build automation scripts and test cases when creating a new battle
			\item R2.9) \textbf{The system} allows \textbf{educators} to set a minimum and maximum number of members for each team of students participating in the battle(s) they want to create
			\item R2.10) \textbf{The system} allows \textbf{educators} to declare whether to require a consolidation stage or not for the battle(s) they want to create
			\item R2.11) \textbf{The system} allows \textbf{educators} to specify what parameters of evaluation (reliability, security...) should be used by the system in order to compute the scores of the students' code solutions through static analysis external tools
			\item R2.12) \textbf{The system} allows \textbf{educators} to set a registration deadline for the battle(s) they want to create
			\item R2.13) \textbf{The system} allows \textbf{educators} to set a submission deadline for the battle(s) they want to create
			\item R2.14-R2.17) are equal to R1.9, R1.11, R1.12, R1.13, they regard the interaction with the GitHub system in order to manage the battles correctly
			\item R2.18) \textbf{The system} gives a time frame to the \textbf{educator} that created a battle to assign personal scores to the students’ code solutions, after the submission deadline of the battle and only if a consolidation stage had been requested at battle creation time
			\item R2.19) At the end of a battle and after the consolidation stage (R13), the system publishes the final rank of all teams that participated in that battle
			\item R2.20) The system automatically updates and shows the rank of students within a tournament
			\item R2.21) When a tournament is closed, the system publishes the rank of all students that participated in the tournament
		\end{itemize}
	\end{itemize}
	
	\subsection{Domain assumptions}
	\begin{itemize}
		\item D1) The GitHub platform is up and running and provides all the functionalities that are expected by the application to work correctly
		\item D2) Student correctly forks the main branch of the GitHub repository in order to store the code solutions
		\item D3) Student correctly writes the automated GitHub workflow to send notification from GitHub to the system every time a commit is performed on the forked branch
		\item D4) Educator correctly writes the build automation scripts for the battles he/she creates on the platform
		\item D6) 
	\end{itemize}
	
	\subsection{Design decisions}
	\subsubsection{Modeling the "consolidation stage"}
	In order to model phenomena, goals and requirements related to the consolidation stage in the RASD, it is first necessary to take a decision at this very early stage on the behavior of the system related to this phase of a battle, since the specification is not very clear on how the application should work. Therefore the following decisions are stated:
	\begin{itemize}
		\item As a precondition, it is taken for granted the fact that only the educator that created the battle is allowed to assign personal scores for that battle (other educators in the tournament do not have permission to do so).
		\item In the page that is displayed to an educator for creating a new battle, there is the possibility to set a simple flag to establish whether a consolidation stage is required at the end of a battle.
		\item Once the battle is over:
			\begin{itemize}
				\item If the consolidation stage was required by the educator when creating the battle, then the application starts a timer in order to impose to the educator a limit on the maximum amount of time available to deliver personal scores. The personal scores of the educator will influence the final ranking only if s/he assigns a score to all the teams by the end of the timer. In case some teams are missing a score, the computation of the final ranking for the battle won't take into account any personal score of the educator.
				\item If the consolidation stage was not required when the battle was created, then the system will be able to compute the final ranking immediately and show it to the participants of the battle.
			\end{itemize}
	\end{itemize}
	These decisions are stated here in order to justify some aspects of the previous model of the system. For instance:
	\begin{itemize}
		\item There is no domain assumption related to the behavior of the educator during the consolidation stage because the system completely controls the process and cannot get stuck in any circumstance (as it would happen in case the system had no timer and waited indefinitely for the educator to input the personal scores)
		\item Some requirements are added 
	\end{itemize}
	