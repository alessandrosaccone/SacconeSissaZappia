\subsection{Scope}
	\app is a software application designed for students and educators in the context of coding challenges, called "code kata battles".
	
	As a first basic feature of the application, \app allows both students and educators to sign up and create a personal account the first time a user accesses the platform. For all subsequent times, the application provides a way to sign in with the credentials chosen in the sign up phase.
	
	Once signed in, educators have the possibility to create new tournaments, setting various parameters while doing it. Every tournament must have a registration deadline, by which the students are asked to subscribe to the tournament if interested. When the tournament is generated on the platform, \app takes care of notifying all the students with an account on the system of the new tournament.
	
	A tournament is composed of various code kata battles that educators can design and publish. The educator creating a new tournament has the possibility of granting the permission of publishing battles to other educators on the platform.
	In order to create a new battle in any tournament, an educator has to follow a series of steps. First off, every battle needs a brief textual description of the problem to be solved. Moreover, the educator has to produce and upload on the platform the test cases and build automation scripts that are going to be employed in order to build the code solutions of the students and verify their correctness.

	Educators are also asked to set some parameters for the battle, such as the registration deadline (by which students can sign up for the battle), the submission deadline (by which solutions for the battle can be handed in), the minimum and maximum number of students per team.
	
	At battle creation time, the application also gives the option to an educator to decide whether or not a "consolidation stage" is necessary for the battle. This is a phase occurring after the submission deadline, during which the educator that created the battle is asked to analyze the students' code solutions and assign a personal score to them. This scores contribute to the final ranking of the battle, along with other parameters. 
	
	Finally, the setup of a new battle requires the educator to establish which criteria the application should employ in order to evaluate the students' code solutions. For instance, valid criteria may be security, reliability, error handling and so on.
	
	Once a battle is published, students that are subscribed to the tournament in which the battle resides are notified by the system of the new coding challenge and are able to join the battle by the registration deadline set by the educator.
	Students can join a battle in two ways: either forming a team with other students or on their own. The platform provides a way to allow a student to invite other students in the same tournament to join a battle together as a team.
	
	Once the registration deadline of a battle passes, \app automatically generates a new GitHub repository dedicated to that battle and sends the link to the remote repository to all the students participating in the battle.
	
	At this point, students are expected to fork the main branch of this repository in order to create a confined environment to store their code solutions. Moreover, students should write an automated GitHub workflow (through GitHub Actions) in order to make the GitHub platform send a notification to \app every time a new commit is added on their branch by any team member. This is a fundamental aspect of the system in order to make everything work correctly.
	At this point, students can write their solutions and push them on their GitHub branch. After a new commit is performed, GitHub should notify the \app system. 
	This notification will cause the \app application to pull the new source code from the specific GitHub branch where the new commits occurred. The platform can then analyze the new code solution in order to calculate the score to assign to it. These calculations are based on a series of factors, such as the number of test cases (provided by the educator) passed, the time went by from the beginning of the battle and the analyses run on the source code with static analysis tools taking into account only the parameters of evaluation specified by the educator at battle creation time (for instance security, reliability...).
	Thanks to this procedure for recalculating the score of the students' solutions, \app will constantly maintain updated the ranking of teams participating in a battle, so that students and educators involved in the coding challenge can see it.
	
	After the submission deadline of a battle, the system will no longer accept any additional solution for that battle. If the consolidation stage was required by the educator during the creation of the battle, \app will start a timer in order to set a specific time frame for the educator to evaluate the students' code solutions. When the timer goes off, if the educator hasn't assessed all the solutions yet, the system won't take into account the personal scores of the educator in order to compute the final ranking. In any case, when the consolidation time frame is over, the system is able to draw the final classification of all teams participating to the battle and publish it. \app will also take care of notifying all students members of these teams of the availability of the ranking.
	
	Different from the ranking of a battle is the ranking of a tournament. Every tournament has a ranking of individual students (not teams). The position of each student in this ranking is the result of the sum of points accumulated by the student in the battles of the tournament the student has participated in.
	\app takes care of automatically updating the tournament ranking every time a battle of the tournament terminates and make it visible to all users of the platform.
	
	The educator that created a tournament is also responsible for closing it. When a tournament gets closed, the final ranking of the tournament is shown and all students who signed up for that tournament are notified by the system.
	
	Briefly taking a look at the application from the student's point of view, it is possible to summarize the functioning of the platform in a  few steps. A student can display the list of all the tournaments available on \app and navigate it to see which ones s/he might be interested in. Then, s/he can subscribe to a tournament and consequently receive notifications by the system of any upcoming battle within that tournament. 
	At this point, the student is able to join any battle in the tournaments s/he's subscribed to, either on his/her own or by inviting other students (always respecting the constraints on the minimum and maximum number of students per team set by the educator who created the battle). After receiving the link of the GitHub repository, students participating in a battle can then fork the main branch of the GitHub repository and start pushing their code solutions there...
	
	Another relevant aspect of the \app system is related to the badges or rewards that are assigned to students participating to a tournament when the tournament ends. At tournament creation time, the educator is able to define these rewards based on specific achievements of students in the tournament, such as the number of commits performed, or the number of battles a student has joined.
	When the tournament is closed, \app calculates the set of students that are eligible for receiving a specific reward and assigns it to them. Every user on the \app platform has a personal profile and on the personal profile of each student, the list of badges earned by the student is made visible to all users of the platform.
	Badges and rewards play a relevant role in \app in order to create a playful environment to make coding problems solutions more enjoyable and entertaining.
\input{\sourcepath 1.2.1WorldMachinePhenomena}