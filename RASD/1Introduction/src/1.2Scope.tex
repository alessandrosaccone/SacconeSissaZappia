\subsection{Scope}
	\app is a software application designed for students and educators in the context of coding challenges, called "code kata battles".
	
	As a first basic feature of the application, \app allows both students and educators to log in the application using their GitHub account. Since all users are required to own a GitHub account in order to successfully interact with the \app platform, the authentication process is delegated to the GitHub system.
	
	Once signed in, educators have the possibility to create new tournaments, setting various parameters while doing it. Every tournament must have a registration deadline, by which the students are asked to subscribe to the tournament if interested. When the tournament is generated on the platform, \app takes care of notifying all the students with an account on the system of the new tournament.
	
	A tournament is composed of various code kata battles that educators can design and publish. The educator creating a new tournament has the possibility of granting the permission of publishing battles to other educators on the platform.
	In order to create a new battle in any tournament, an educator has to follow a series of steps. First off, every battle needs a brief textual description of the problem to be solved. Moreover, the educator has to produce and upload on the platform the test cases and build automation scripts that are going to be employed in order to build the code solutions of the students and verify their correctness.

	Educators are also asked to set some parameters for the battle, such as the registration deadline (by which students can sign up for the battle), the submission deadline (by which solutions for the battle can be handed in), the minimum and maximum number of students per team.
	
	At battle creation time, the application also gives the option to an educator to decide whether or not a "consolidation stage" is necessary for the battle. This is a phase occurring after the submission deadline, during which the educator that created the battle is asked to analyze the students' code solutions and assign a personal score to them. This scores contribute to the final ranking of the battle, along with other parameters. 
	
	Finally, the setup of a new battle requires the educator to establish which criteria the application should employ in order to evaluate the students' code solutions. For instance, valid criteria may be security, reliability, error handling and so on.
	
	Once a battle is published, students that are subscribed to the tournament in which the battle resides are notified by the system of the new coding challenge and are able to join the battle by the registration deadline set by the educator.
	Students can join a battle in two ways: either forming a team with other students or on their own. The platform provides a way to allow a student to invite other students in the same tournament to join a battle together as a team.
	
	Once the registration deadline of a battle passes, \app automatically generates a new GitHub repository dedicated to that battle and sends the link to the remote repository to all the students participating in the battle.
	
	At this point, students are expected to fork the main branch of this repository in order to create a confined environment to store their code solutions. Moreover, students should write an automated GitHub workflow (through GitHub Actions) in order to make the GitHub platform send a notification to \app every time a new commit is added on their branch by any team member. This is a fundamental aspect of the system in order to make everything work correctly.
	At this point, students can write their solutions and push them on their GitHub branch. After a new commit is performed, GitHub should notify the \app system. 
	This notification will cause the \app application to pull the new source code from the specific GitHub branch where the new commits occurred. The platform can then analyze the new code solution in order to calculate the score to assign to it. These calculations are based on a series of factors, such as the number of test cases (provided by the educator) passed, the time went by from the beginning of the battle and the analyses run on the source code with static analysis tools taking into account only the parameters of evaluation specified by the educator at battle creation time (for instance security, reliability...).
	Thanks to this procedure for recalculating the score of the students' solutions, \app will constantly maintain updated the ranking of teams participating in a battle, so that students and educators involved in the coding challenge can see it.
	
	After the submission deadline of a battle, the system will no longer accept any additional solution for that battle. If the consolidation stage was required by the educator during the creation of the battle, \app will start a timer in order to set a specific time frame for the educator to evaluate the students' code solutions. When the timer goes off, if the educator hasn't assessed all the solutions yet, the system won't take into account the personal scores of the educator in order to compute the final ranking. In any case, when the consolidation time frame is over, the system is able to draw the final classification of all teams participating to the battle and publish it. \app will also take care of notifying all students members of these teams of the availability of the ranking.
	
	Different from the ranking of a battle is the ranking of a tournament. Every tournament has a ranking of individual students (not teams). The position of each student in this ranking is the result of the sum of points accumulated by the student in the battles of the tournament the student has participated in.
	\app takes care of automatically updating the tournament ranking every time a battle of the tournament terminates and make it visible to all users of the platform.
	
	The educator that created a tournament is also responsible for closing it. When a tournament gets closed, the final ranking of the tournament is shown and all students who signed up for that tournament are notified by the system.
	
	Briefly taking a look at the application from the student's point of view, it is possible to summarize the functioning of the platform in a  few steps. A student can display the list of all the tournaments available on \app and navigate it to see which ones s/he might be interested in. Then, s/he can subscribe to a tournament and consequently receive notifications by the system of any upcoming battle within that tournament. 
	At this point, the student is able to join any battle in the tournaments s/he's subscribed to, either on his/her own or by inviting other students (always respecting the constraints on the minimum and maximum number of students per team set by the educator who created the battle). After receiving the link of the GitHub repository, students participating in a battle can then fork the main branch of the GitHub repository and start pushing their code solutions there...
	
	Another relevant aspect of the \app system is related to the badges or rewards that are assigned to students participating to a tournament when the tournament ends. At tournament creation time, the educator is able to define these rewards based on specific achievements of students in the tournament, such as the number of commits performed, or the number of battles a student has joined.
	When the tournament is closed, \app calculates the set of students that are eligible for receiving a specific reward and assigns it to them. Every user on the \app platform has a personal profile and on the personal profile of each student, the list of badges earned by the student is made visible to all users of the platform.
	Badges and rewards play a relevant role in \app in order to create a playful environment to make coding problems solutions more enjoyable and entertaining.
\subsubsection{World and Machine phenomena}
	This section summarizes the previous description into lists of phenomena (events) that occur in the world of interest for the system to be developed. Phenomena have to be interpreted simply as events occurring in the whole domain that is being analyzed in the document, so they have been stripped of any constraint (that will be better specified in the requirements section). 
    \newline Phenomena can be divided into:
	\begin{itemize}
		\item \textbf{World phenomena}: events happening outside the system and on which the system has no control.
		\item \textbf{Machine phenomena}: events happening internally in the system, independent from the outside world.
		\item \textbf{Shared phenomena}: events that are have an influence on both the system and the world surrounding it. Usually they are further split into two classes:
		\begin{itemize}
			\item \textbf{Machine controlled shared phenomena}: events triggered or initiated by the system with a relevant impact in the domain in which the system works.
			\item \textbf{World controlled shared phenomena}: events initiated by entities of the world that are impactful for the system.
		\end{itemize}
	\end{itemize}
	
	\paragraph{World phenomena}
	\begin{itemize}
		\item WP1) \textbf{User} has a personal GitHub account.
		\item WP2) \textbf{Educator} wants to support students in bettering their software development skills.
		\item WP3) \textbf{Educator} comes up with new code kata battles to publish on the system.
		\item WP4) \textbf{Educator} writes the textual description of a new battle s/he wants to publish.
		\item WP5) \textbf{Educator} designs the test cases and the build automation scripts of a new battle s/he wants to publish.
		\item WP6) \textbf{Student} wants to improve their software development skills.
		\item WP7) \textbf{Student} forks the main GitHub repository that hosts a battle to upload 	his/her code solutions.
		\item WP8) \textbf{Student} writes an automated workflow through GitHub Actions to setup the interaction between GitHub and the system.
		\item WP9) \textbf{Student} works at developing the code solutions for a battle.
		\item WP10) \textbf{Student} pushes on the forked branch of the GitHub repository hosting a 	battle the code solutions, generating new commits.
	\end{itemize}
	
	\paragraph{Machine phenomena}
	\begin{itemize}
		\item MP1) \textbf{The system} pulls new code solutions from GitHub.
		\item MP2) \textbf{The system} runs build automation scripts and test cases on new code solutions.
		\item MP3) \textbf{The system} calculates the percentage of test cases passed by a new code solution pulled from a GitHub repository.
		\item MP4) \textbf{The system} calculates the amount of time passed between the beginning of a battle and the moment in which a new submission has been submitted by a student through GitHub.
		\item MP5) \textbf{The system} runs static analysis on the new code solutions exploiting external static analysis tools.
		\item MP6) \textbf{The system} calculates the score to be assigned to a new code solution.
		\item MP7) \textbf{The system} computes the partial or final ranking of students for a tournament.
		\item MP8) \textbf{The system} computes the partial or final ranking of teams for a battle.
		\item MP9) \textbf{the system} computes the set of students that are eligible for receiving a badge (reward) at the end of a tournament.
	\end{itemize}
	
	\paragraph{World controlled shared phenomena}
	\begin{itemize}
		\item WSP1) \textbf{User} logs in the system using his personal GitHub account.
		\item WSP2) \textbf{Educator} creates a new tournament.
		\item WPS3) \textbf{Educator} sets the title of the tournament s/he wants to create.
		\item WSP4) \textbf{Educator} defines the badges (rewards) for the tournament s/he wants to create.
		\item WSP5) \textbf{Educator} grants permissions to create battles to other educators.
		\item WSP6) \textbf{Educator} creates a battle in a tournament.
		\item WPS7) \textbf{Educator} sets the title of the battle s/he wants to create.
		\item WSP8) \textbf{Educator} uploads on the system the textual description of the battle s/he wants to create.
		\item WSP9) \textbf{Educator} uploads on the system the test cases and build automation scripts for the battle s/he wants to create.
		\item WSP10) \textbf{Educator} inputs in the system the minimum and maximum number of students per group allowed for the battle s/he wants to create.
		\item WS11) \textbf{Educator} inputs in the system the registration deadline of the battle s/he wants to create.
		\item WSP12) \textbf{Educator} inputs in the system the submission deadline for the students’ solutions of the battle s/he wants to create.
		\item WSP13) \textbf{Educator} specifies to the system whether a consolidation stage is required after the submission deadline of the battle s/he wants to create.
		\item WSP14) \textbf{Educator} selects which aspects (reliability, maintainability, security...) should be taken into account by the system when computing the score of the students' code solutions for the battle s/he wants to create.
		\item WSP15) \textbf{Educator} checks the code solutions of the teams participating in a battle and assigns a personal score to each one of them.
		\item WSP16) \textbf{Educator} closes a tournament.
		\item WSP17) \textbf{User} opens the description page of a tournament.
		\item WSP18) \textbf{Student} subscribes to a tournament.
		\item WSP19) \textbf{User} opens the description page of a battle.
		\item WSP20) \textbf{Student} sends invitations to other students to ask to join a battle together as a team.
		\item WSP21) \textbf{Student} joins a battle of a tournament either forming a group with other students or on his/her own.
		\item WSP22) \textbf{User} opens his/her personal profile on the platform.
		\item WSP23) \textbf{User} searches another user by name on the platform to see his/her personal profile.
		\item WSP24) \textbf{User} opens the personal profile of another user on the platform.
		\item WSP25) \textbf{User} opens the list of badges s/he own or of the ones owned by another user on the platform.
		\item WSP26) \textbf{User} opens the partial or final ranking of students in a tournament.
		\item WSP27) \textbf{User} opens the partial or final ranking of teams in a battle.
		\item WSP28) \textbf{GitHub} notifies the system every time a new commit is pushed on a repository that stores the code solutions for a battle.
	\end{itemize}
	
	\paragraph{Machine controlled shared phenomena}
	\begin{itemize}
		\item MSP1) \textbf{The system} shows to a \textbf{user} the interface to log in the platform using his/her personal GitHub account.
		\item MSP2) \textbf{The system} shows to a \textbf{user} the initial page (home page) of the application.
		\item MSP3) \textbf{The system} shows to an \textbf{educator} the interface for creating a new tournament.
		\item MSP4) \textbf{The system} shows to an \textbf{educator} the interface for creating a new battle in an ongoing tournament.
		\item MSP5) \textbf{The system} shows to an \textbf{educator} the interface to assign personal scores during the consolidation stage of a battle.
		\item MSP6) \textbf{The system} shows to a \textbf{user} the list of tournaments available on the platform.
		\item MSP7) \textbf{The system} shows to a \textbf{user} the description page of a tournament.
		\item MSP8) \textbf{The system} shows to a \textbf{student} the interface to subscribe to a tournament.
		\item MSP9) \textbf{The system} shows to a \textbf{user} the list of battles within a tournament.
		\item MSP10) \textbf{The system} shows to a \textbf{user} the description page of a battle.
		\item MSP11) \textbf{The system} shows to a \textbf{student} the interface to invite other students to join him/her in a battle as a team.
		\item MSP12) \textbf{The system} shows to a \textbf{student} the interface to join a battle.
		\item MSP13) \textbf{The system} shows to a \textbf{user} his personal profile.
		\item MSP14) \textbf{The system} shows to a \textbf{user} the personal profile of another user on the platform.
		\item MSP15) \textbf{The system} shows to a \textbf{user} the list of badges owned by a student on the platform.
		\item MSP16) \textbf{The system} shows to a \textbf{user} the partial and final ranking of students in a tournament.
		\item MSP17) \textbf{The system} shows to a \textbf{user} the partial and final ranking of teams in a battle.
		\item MSP18) \textbf{The system} notifies all \textbf{students} subscribed to the platform when a new tournament is created.
		\item MSP19) \textbf{The system} notifies all \textbf{students} subscribed to a tournament when a new battle is created in the tournament.
		\item MSP20) \textbf{The system} creates a new \textbf{GitHub} repository dedicated to a battle.
		\item MSP21) \textbf{The system} sends the link of the GitHub repository to all the \textbf{students} subscribed to the battle.
		\item MSP22) \textbf{The system} shows and keeps updated the scores assigned to the students' code solutions for a battle.
		\item MSP23) \textbf{The system} notifies all \textbf{students} that participated in a battle when the final ranking of the battle is available.
		\item MSP24) \textbf{The system} notifies all \textbf{students} subscribed to a tournament when the tournament is closed.
		\item MSP25) \textbf{The system} assigns badges (rewards) to \textbf{students} when a tournament ends.
	\end{itemize}