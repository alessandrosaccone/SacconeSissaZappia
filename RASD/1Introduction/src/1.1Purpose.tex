\subsection{Purpose}
The main idea behind the \app platform comes from the term "kata", which is a Japanese word describing detailed patterns of movements repeated many times in order to memorize and master them. This kind of exercise is usually applied in learning karate. \app aims at exporting the same learning process also in the domain of coding and programming. 
 
The software described in this document involves two main actors, which are students and educators.  Educators can propose code kata battles (or simply battles for brevity), which are problems and teasers to be addressed by coding with a specific programming language,  while students can engage in these battles in order to find solutions to them. Code kata battles are published in the context of tournaments in which students compete with each other.
 
The main goals of this system are tightly related to the users of the application. On the educator's side, \app allows to easily organize tournaments and coding battles, also for a great number of students. On the student's side, \app creates an enjoyable and playful environment to foster and improve the student's coding skills.
 
It is possible to summarize the most important objectives of the software to be developed in the following list of goals:
\begin{itemize}
	\item G1) The system allows students to practice and improve their coding skills by developing solutions to code kata battles.
	\item G2) The system provides to educators a platform to publish code kata battles and easily manage tournaments and battles for students.
	\item G3) The system makes the task of coding battles' solutions as fun as possible, introducing elements like competitiveness, rankings, teamwork and badges (or rewards).
\end{itemize}

\newpage