\subsubsection{Scenarios}
\textbf{SCENARIO 1 - Educator logs in the system} \\
    Emanuele is a professor at Politecnico di Milano in the Computer Science department. He has just discovered a new application called \app that allows educators to organize tournaments of students to make them compete on coding battles.\\
    Emanuele is very enthusiast about the idea because he thinks it might be a great opportunity for his students to better their software development and problem solving skills. He therefore opens \app on his laptop. The first interface showing up is the log in interface, in which the application asks Emanuele to sign in using his GitHub personal account. He clicks on the button to be redirected to the GitHub authentication page and uses his credentials as required. GitHub calls back the \app platform and now Emanuele is logged in and can start experimenting with this brand new system.\\
    
    \textbf{SCENARIO 2 - Student logs in the system} \\
    Matteo is a student enrolled in a Computer Science master's degree at the University of Barcelona. He's passionate about coding and software development in general. One day, he comes across a new app called \app that allows students to compete against each other in coding battles organized by educators from many universities spread across the world.  \\
    This application immediately catches his attention, so he decides to open it on his laptop. The first interface showing up is the log in interface, in which the \app platform asks Matteo to sign in with his personal GitHub account. Matteo clicks on the button that redirects to the GitHub authentication page and logs in as requested.
    Now, he's signed in the platform and can start playing around with it to see if he likes it.\\
    
    \textbf{SCENARIO 3 - Educator creates a new tournament}\\
    Mario is a renowned university professor in the field of Software Engineering. He has been using the \app application for a while now, but he's never organized a tournament on this platform.
    One day, he decides to create his first tournament on \app. So, he opens the application and signs in. Among the various buttons that are available on the educators' home page of the platform, he clicks on the one to create a new tournament. As a consequence, \app opens up a window in which various parameters can be set for the new tournament. Mario can set the name of the tournament, a description of it and the registration deadline by which students are asked to subscribe to the tournament if interested. Moreover, there is a section dedicated to the definition of badges (rewards) to be assigned to students at the end of the tournament based on some goals or achievements. Mario can declare on \app what kind of rewards he wants allocate and for each reward the achievement(s) that have to be accomplished by a student in order to earn the badge.\\
    After filling in the all fields of the form, Mario can confirm the creation of the tournament and the application will show him the newly-created tournament page.
    \app will also send a notification to all the student on the platform informing them of the new available tournament.\\

    \textbf{SCENARIO 4 - Educator creates a new battle}\\
    Alex is a professor at the University of London specialized in artificial intelligence. Lately, he's been using the \app platform to help his students practice on some algorithms that he's explaining during the lectures. One morning, he comes up with a new idea for a code kata battle that may be very beneficial for the students that are going to take his exam in the following months. Thus, he seats at the desk in his bedroom and starts writing a textual description of the battle, along with the build automation scripts and test cases that are required by \app in order to build and test the students' solutions to the exercise. Once done with these tasks, Alex takes his laptop and opens the \app application. He logs in through his GitHub account and clicks from the home page the button to create a new battle. A new interface pops up displaying all the tournaments in which Alex has permissions to publish battles (so the tournaments Alex created and the ones other educators granted Alex permissions on). Alex selects one of these tournaments and the form to set the battle's parameters and characteristics shows up. Through this interface, Alex is able to upload the textual description, build automation scripts and test cases that he previously designed; he also set the battle's name, the registration and submission deadlines, the minimum and maximum number of students per team allowed. Moreover, the aspects on which \app has to base its automatic evaluations of the students' solutions (reliability, maintainability...) can be defined and it is possible to establish whether to require a consolidation stage at the end of the battle (in order to allow Alex to assign personal scores).
    Once all of these parameters have been set, Alex clicks on the confirmation button and the page dedicated to the newly created battle pops up on his display.
    Immediately after this, \app will fire a notification message to all the students subscribed to Alex's tournament in order to inform them of the new available battle.\\

	\textbf{SCENARIO 5 - Educator grants the permission to publish battles in his/her tournament to another educator}\\
	George is a professor at MIT's Software Engineering department. Lately, he's been actively using the \app software because he initiated a tournament of coding challenges for students centered around the Python programming language. He's published many battles over the last week, and that's why he's running out of ideas for new problems to propose. That's why he decides to ask for some help from one of his closest colleagues, Laura. In order to allow Laura to publish battles inside George's tournament, George has to grant her permissions to do so. Thus, George opens the \app application and clicks on the button to display the tournaments he created. Among these, he selects the one in which he wants Laura to help him. The home page of the selected tournament pops up and there is a button to grant publishing permissions to another educator. Once clicked, George is able to search for Laura's account through her username on the platform and then send her a request to be added as an educator with publishing permissions inside his tournament. Once these steps have been carried out, Laura is able to see George's request on her profile and accept it. George's tournament will then appear in the list of tournaments in which she can create battles. Now George will no longer be the only one keeping the tournament alive with new battles.\\

    \textbf{SCENARIO 6 - Student subscribes to a tournament}\\
    Alessandro is an artificial intelligence student at the Technical University of Munich. It has been a while since the last time he coded a program of any kind, so he feels a little bit rusty on that. He would like to brush up on this skill and he decides to use the new app \app for this purpose.
    After logging in the system, \app displays all the available tournaments on the platform. Thus, Alessandro starts browsing this list, searching for something that he might find interesting. 
    At some point, he stumbles upon a tournament created by one of his favorite professors at his university, so he clicks on it and the home page of the tournament pops up. George clicks on the button to subscribe and confirms his choice. Consequently, \app moves the selected tournament in the list of tournaments Alessandro is subscribed to.
    From this moment on, Alessandro will be notified by the \app app every time a new battle will be published inside the tournament.\\

    \textbf{SCENARIO 7- Student joins a battle on his/her own (without a team)}\\
    Lorenzo is a student at the University of Pisa specialized in data management. He's been eagerly trying to reach the first position in a tournament of the \app application, but he's still some points behind the first in the ranking. He doesn't want to give up, so he opens \app and navigates to the tournament in which he's competing. In order to get additional points he has to join another battle. So he commences the search for the battle, reading some descriptions of the battles published in the tournament whose registration deadline hasn't passed yet. He eventually finds one that might be suitable for him, so he clicks on the button to join it. \app promptly shows Lorenzo the description page of the battle, where there's a button to join it. Once clicked, an interface pops up in which it is possible to select whether to join the battle as a single player or with other students in a team. Since Lorenzo usually prefers to work on his own, he opts for the single player and confirms.
    The battle is therefore added to the list of battles Lorenzo is participating in. Lorenzo will be able to submit his code solutions when the registration deadline passes and \app will have created the GitHub repository dedicated to the battle.\\

    \textbf{SCENARIO 8 - Student joins a battle with other students as a team}\\
    Lucia is a student at the Politecnico di Torino university and she's enrolled in the cybersecurity program. Her group of university friends have been talking incessantly about creating a team on the new \app application to solve together a coding battle proposed by one of their professor specialized in cryptography called Adrian. She lets her friends persuade her to do that, so she opens the \app application on her laptop and navigates to the tournament that contains the interesting battle. She and her friends are already subscribed to such tournament, since it is a very popular one organized inside the university, therefore she doesn't have to sign up for that. Then, Lucia spots the coding challenge her friends were talking about and clicks on it. Since the registration deadline hasn't passed yet, \app promptly shows the interface to join the battle, in which it is possible to select whether to sign up as a single player or with other students as a team. Obviously she picks the latter option and a new window pops up, in which Lucia has the possibility to search by username her friends on the platform and send them a request to join the battle in her team. Once this process is carried out, Lucia confirms and the \app application displays the entry page of the battle. The battle is therefore added to the list of battles Lucia is participating in.
    Lucia's friends will be able to take part in the battle as soon as they accept the request sent by the system. At that point, the battle will also be added by \app to the list of battles in which they are competing.
    Lucia and her friends will be able to submit their code solutions as soon as the registration deadline for the battle passes and \app creates the dedicated GitHub repository.\\
    
    \textbf{SCENARIO 9 - Student receives the link of the GitHub repository dedicated to a battle and starts pushing his code solutions on the forked branch of the GitHub repository}\\
    Francesco is a student at the University of Edinburgh where he studies computer engineering. A couple of days ago he joined a new battle on the \app application which was very appealing to him as it was about optimization algorithms. He has been very busy with some commitments that he had taken, so he hasn't opened \app since the subscription to the battle. Luckily, at around midday, he receives a notification from the app saying that the battle was open (since the registration deadline had passed) and that it was possible to hand in the code solutions to the challenge on a forked branch of the dedicated GitHub repository. The link to the GitHub repository was also attached to the notification message.
    This alert from \app suddenly reminds Francesco of the battle. So he runs back home from the university and first of all sets up the GitHub environment. He forks the main branch of the GitHub repository dedicated to the battle and also writes an automated GitHub workflow with GitHub Actions in order to fire a notification from GitHub to the \app platform every time he pushes a new code solution to the forked branch.
    At this point, Francesco is able to start pushing his solutions to the forked branch on GitHub and see the corresponding scores on \app. So he writes a code block that might pass the test cases of the battle and submits it on his GitHub branch.\\
    GitHub will immediately fire a notification to the \app platform. As a consequence, \app downloads Francesco's code solution and uses the build automation scripts provided by the educator that published the battle in order to build the code. At this point, if \app is not even able to build the code solution, an error message is returned to Francesco stating the problem. In the other situation, the test cases (provided by the educator that created the battle) are employed to verify the correctness of the solution. Moreover, \app exploits some external static analysis tool to rate some other parameters of the solution (chosen by the educator that created the battle) like security, maintainability and so on. These pieces of information are put together with the time passed since the beginning of the battle in order to assign a score to Francesco's solution.
    In a few moments after the submission of the solution, Francesco notices on the \app platform the score assigned to his solution and also his updated position in the battle ranking, based on the new score.\\
  
    \textbf{SCENARIO 10 - Educator manually evaluates the students' code solutions of one of his/her battles during the consolidation stage}\\
    Michele is a meticulous computer science teacher in a little high school in Rome. Lately, he's been publishing several battles on the \app platform for his students to let them exercise on some new algorithms he explained during the lectures. Since he wants to personally assess the code that his students write, he always specifies at battle creation time that a consolidation stage is required for his battles. Today, Michele knows that the submission deadline of one of his battles will pass at 6pm, and that after that time, the \app will allow him to assign a personal evaluation to his students' solutions. At 6:30pm, when Michele comes home from work, he makes himself a cup of tea and opens the \app platform. In the list of battles that he created, he clicks on the one that requires the consolidation stage to be carried out. A new interface shows up, where the list of teams subscribed to the battle is shown and for each team there is a field on the side where it is possible to type in the score Michele wants to assign. Within this interface, there is also a timer that specifies how much time Michele has to assess the code solutions. After the timer goes off, if Michele hasn't provided a score for each team in the battle, the consolidation stage won't be taken into account by \app in order to compute the final ranking of the battle.
    All the source code for the solutions is available for Michele on the GitHub platform, so he can review it all and complete the consolidation stage for the battle by the specified time frame.
    When the timer goes off, \app is able to draw the final ranking of teams for the battle and publish it on the platform. \app will also send a notification to all the participants of the battle to notify them of the available hierarchy of teams.\\

	\textbf{SCENARIO 11 - A battle terminates and the ranking of teams is published}\\
	Clara, a data analytics student enrolled in a Master's program at ETH Zurich university is impatiently waiting for the end of a battle on the \app platform. She's worked really hard over the last weekend in order to reach the first position in the ranking with a solution that scored 95/100. At 1pm, the submission deadline for the battle has finally passed. Since no consolidation stage was required by the professor who published the battle, \app immediately calculates the total ranking of teams and fires a notification to all the students participating in the battle. Also Clara receives the alert stating that the battle was over and that the final ranking was available on the app. She enthusiastically opens \app on her laptop just to find out that during the last 10 minutes before the submission deadline, another competitor handed in a solution that received a score higher than hers.\\ 

	\textbf{SCENARIO 12 - Educator closes a tournament that s/he previously created}\\
    Olivia is a famous professor in the field of Bioinformatics. Over a month ago, she decided to create a new tournament on the \app platform in which she's been started publishing coding battles that are biology-themed. Many students from all over the city in which Olivia lives liked the idea of this tournament and decided to take part in it. Now, since the tournament has been going on for over a month, the challenges are becoming more and more repetitive, therefore Olivia decides to officially close it. She opens the \app application on her laptop and logs in. In the home page, she clicks on a button to display the list of tournaments that she created. She selects the tournament she wants to close, clicks the button to close the tournament and confirms her choice. The \app platform immediately sends a notification to all the students subscribed to the tournament informing them that the tournament is now closed and the final ranking of students is available. Since \app always maintains updated the ranking of students in a tournament, there is no need to recalculate the ranking here, just to display it. Moreover, since Olivia defined some badges at tournament creation time, \app evaluates the sets of students that are eligible to receive these badges and assigns them accordingly. The personal profile of students that received a reward from the tournament is automatically updated by the app.\\

    \textbf{SCENARIO 13 - User opens the ranking of a tournament on the \app platform}\\
    Christian is a student at Princeton University where he specializes in game design and development. He knows that some of his classmates are participating in a tournament on the \app platform organized by some of the professor of the engineering department and he's very curious about the ranking of this tournament. So, he opens \app, logs in with his personal GitHub account and in the home page that is displayed right after the log in he can see all the tournaments available on \app. He scrolls a little bit down and eventually finds the tournament he was looking for. So he clicks on it and in the dedicated page, the total ranking of students is shown, updated with the very last scores assigned to the students' solutions. Christian can finally see how his classmates are positioned.\\

    \textbf{SCENARIO 14 - User opens the ranking of a battle s/he is involved in}\\
     Emily is a web development student at the University of Toronto. She's been working with her team on the solution for a battle on the \app platform for a while now, but she cannot get a score higher than 55/100. Since she doesn't know if that is going to be enough to reach the top 10 positions of the battle ranking, she decides to give a look at the partial ranking for the competition, to have an idea of how well the other teams are doing so far. So she opens the \app app on her laptop and navigates to the list of tournaments she's subscribed to. Once she's selected the correct tournament, she can see the list of battles belonging to that tournament that she signed up for and among them she picks the one she wants to see the ranking of. Consequently, \app checks whether Emily is actually involved in the battle or not and displays the ranking only in the affirmative case. 
     At the end, Emily discovers that she and her team are in the very last position!\\

    \textbf{SCENARIO 15 - User searches by username the personal profile of another user on the platform and looks at his/her badges}\\
    Filippo, a student at the University of Bologna, where he's conducting some important research on Cloud Computing, has been using the \app application to compete with other classmates on coding challenges. Since he's very competitive, he wants to give a look at his friends' profiles on the platform to check if they have obtained more badges than him. So he opens the \app application on his laptop, logs in and clicks on the search box in the home page to type in the username of one of his friends. Once he finds the correct user on the application, he clicks on it and \app displays the personal profile of this user. Filippo is able now to inspect the set of badges earned by his friend and compare them with the ones he obtained so far.\\